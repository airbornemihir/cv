% LaTeX file for resume 
% This file uses the resume document class (res.cls)

\documentclass[margin]{res} 
% the margin option causes section titles to appear to the left of body text 
\textwidth=5.2in % increase textwidth to get smaller right margin
%\usepackage{helvetica} % uses helvetica postscript font (download helvetica.sty)
%\usepackage{newcent}   % uses new century schoolbook postscript font 

\begin{document} 
 
\name{Mihir P Mehta\\[12pt]} % the \\[12pt] adds a blank line after name
 
\address{Department of Computer Science \\University of Texas at Austin \\ Austin, TX 78712  \\
        +1 512-952-0104 \\ mihir@cs.utexas.edu }

 
\begin{resume} 
 
\section{Areas of interest} 
Formal verification, programming languages, operating systems.

\section{Education} 
{\bf Ph.D. in Computer Science}, University of Texas at Austin. \hfill (2014 - present)\\
GPA: 3.83/4 \hfill (Spring 2015)\\
{\bf B.Tech. in Computer Science and Engineering (CSE)}, Indian Institute of
Technology (IIT) Delhi. \hfill (2009 - 2013)\\
GPA: 7.9/10\\
{\bf Exchange semester}, Ecole des Mines, Saint-Etienne. \hfill (2011)

\section{Experience}

 {\bf Research Assistant,} With Professors Isil Dillig and Thomas
 Dillig, CS department, UT Austin. \hfill (2014-2015)
 \begin{itemize} \itemsep -2pt  % reduce space between items
 \item Worked on program verification in object-oriented languages.
 \item Coded up a prototype verifier based on Hoare logic and weakest
   pre-conditions using the Soot compiler framework for Java.
 \end{itemize}
 
 {\bf Software Engineer,} Samsung Research Institute, Noida, India. \hfill (2013-2014)
 \begin{itemize} \itemsep -2pt  % reduce space between items
 \item Worked as a researcher in Samsung's Systems Core Group.
 \item Primarily tasked with optimising the Linux kernel for Samsung's
   Android devices.
 \item Improved core components of the Linux virtual memory subsystem.
 \end{itemize}

\section{Undergraduate Thesis} 
        {\bf Algorithms for prebisimilarity} With Professor S
        Arun Kumar, CSE Department, IIT Delhi    \hfill
        (2012-2013) 
        \begin{itemize} \itemsep -2pt
        \item Conceptualised and implemented a toolkit for
          verifying bisimilarity and other properties of timed automata
          and labelled transition systems.
        \item Leveraged UPPAAL model checker to add support
          for difference bound matrices.
        \item Improved an algorithm for generating a zone
          graph from a timed automaton.

	\end{itemize}

\section{Coursework}
        {\bf UT Austin}
        \begin{itemize} \itemsep -2pt
        \item CS395T Automated Logical Reasoning \hfill (Spring 2015)
        \item CS388L Introduction to Mathematical Logic \hfill (Spring 2015)
        \item CS388S Formal Verification and Semantics \hfill (Fall
          2014)
        \item CS395T Automatic Verification of Software \hfill (Fall
          2014)
        \end{itemize}

        {\bf IIT Delhi (graduate courses only)}
        \begin{itemize} \itemsep -2pt
        \item CSL728 Compiler Design \hfill (Fall 2012)
        \item CSL705 Theory of Computation \hfill (Spring 2012)
        \item MAL704 Numerical Optimisation \hfill (Spring 2012)
        \end{itemize}

\section{Scholastic Achievements}
 \begin{itemize} \itemsep -2pt  % reduce space between items
 \item Awarded the UT Austin Graduate School's College Recruitment
   Fellowship. \hfill (2014-2017)
 \item Secured All India Rank 138 in the Joint Entrance Examination
   (IIT-JEE) among 400000 candidates. \hfill (2009)
 \item Secured All India Rank 29 in the All India Engineering Entrance
   Examination (AIEEE) among 1000000 candidates.\hfill (2009)
 \item Scored 99 percentile in Verbal and Analytical Reasoning,
   GRE. \hfill (2012)
 \end{itemize}

% Tabulate Computer Skills; p{3in} defines paragraph 3 inches wide
\section{Technical \\ Skills}
   \begin{tabular}{l p{3in}}
    \underline{Programming languages:} Functional languages (OCaml,
    SML), \\ logic programming languages (Prolog),
    hardware description languages (VHDL). \\

    \underline{Operating systems:} GNU/Linux (application development
    and kernel development). \\

    \underline{Compiler frameworks:} Soot (Java), LLVM (C++). \\

    \underline{Others:} Xilinx, Matlab, PostgreSQL. \\
 \end{tabular}

\section{Others}
\begin{tabular}{l p{3in}}
  \underline{Languages:} English, French, Gujarati, Hindi. \\
  \underline{Outside of work:} I enjoy reading, blogging, swimming,
  strength training, and playing my guitar.
\end{tabular}

\end{resume} 
\end{document} 
