% LaTeX file for resume
% This file uses the resume document class (res.cls)

\documentclass[margin]{res}
% the margin option causes section titles to appear to the left of body text
\textwidth=5.2in % increase textwidth to get smaller right margin
%\usepackage{helvetica} % uses helvetica postscript font (download helvetica.sty)
%\usepackage{newcent}   % uses new century schoolbook postscript font

\usepackage{url}

\begin{document}

\name{Mihir P Mehta\\[12pt]} % the \\[12pt] adds a blank line after name

\address{Department of Computer Science \\University of Texas at Austin }

\address{+1 512-952-0104 \\ \url{www.cs.utexas.edu/~mihir} }


\begin{resume}

%% \section{Areas of interest}
%% Formal verification, programming languages, operating systems.

\section{Education}
{\bf Ph.D., Computer Science}, University of Texas at Austin. \hfill (2014 - present)\\
GPA: 3.3/4 \hfill (Spring 2018)\\
{\bf B.Tech., Computer Science and Engineering}, Indian Institute of
Technology (IIT) Delhi. \hfill (2009 - 2013)\\
GPA: 7.9/10\\
{\bf Exchange semester}, Ecole des Mines, Saint-Etienne. \hfill (2011)

\section{Professional Experience}

 {\bf Research Intern} at Oracle Corp., Belmont, CA, USA. \hfill (2018)
 \begin{itemize} \itemsep -2pt  % reduce space between items
 \item Completed a code proof to certify the correctness of a highly
   optimised assembly language program.
 \item Contributed to a timing analysis of this program, to ensure the
   avoidance of race conditions.
 \item Studied the potential use of the TLAPS theorem prover for
   distributed systems, and created some preliminary internal
   documentation.
 \end{itemize}

 {\bf Research Intern} at Apple Computer, Inc., Austin, TX, USA. \hfill (2017)
 \begin{itemize} \itemsep -2pt  % reduce space between items
 \item Used model checking tools towards verifying Apple's hardware microarchitectures
 \item Developed proofs of correctness of hardware components with
   respect to specifications, with code changes where necessary.
 \end{itemize}

 {\bf Research Intern} at Intel Corporation, Austin, TX, USA. \hfill (2015)
 \begin{itemize} \itemsep -2pt  % reduce space between items
 \item Built a Pintool to dynamically analyse executables.
 \item Augmented
   the analysis with fine-grained information obtained from static
   analysis techniques.
 \end{itemize}

 {\bf Software Engineer} at Samsung Research Institute, Noida, India. \hfill (2013-2014)
 \begin{itemize} \itemsep -2pt  % reduce space between items
 \item Optimised the Linux kernel for Samsung's
   Android devices.
 \item Improved core components of the Linux virtual memory subsystem.
 \end{itemize}

\section{Research Experience}

 {\bf Filesystem modelling for FAT32} with Professor William R. Cook, CS department, UT Austin. \hfill (2016-present)
 \begin{itemize} \itemsep -2pt  % reduce space between items
 \item Developed a binary-compatible executable model for the FAT32 file system.
 \item Used the model as a basis for separation-based reasoning about
   filesystems and file-manipulating programs with ACL2.
 \item Published papers on this work in the proceedings of
   ACL2-2018 and ITP-2019.
 \end{itemize}

 {\bf Program verification in object-oriented languages} with Professors Isil Dillig and Thomas
 Dillig, CS department, UT Austin. \hfill (2014-2015)
 \begin{itemize} \itemsep -2pt  % reduce space between items
 \item Developed a prototype verifier based on Hoare logic and weakest
   pre-conditions.
 \item Used the Soot compiler framework to generate
   verification conditions and the Z3 theorem prover to discharge them.
 \item Generated example
   inputs demonstrating bugs in several test programs.
 \end{itemize}

 {\bf Algorithms for bisimilarity} with Professor S
 Arun Kumar, CSE Department, IIT Delhi    \hfill
 (2012-2013)
 \begin{itemize} \itemsep -2pt
 \item Conceptualised and implemented a toolkit for
   verifying bisimilarity and other properties of timed automata
   and labelled transition systems.
 \item Improved an algorithm for generating a zone
   graph from a timed automaton.

 \end{itemize}

\section{Publications}
 {\bf Formalising Filesystems in the ACL2 Theorem Prover: an
   Application to FAT32}. In: \textit{Proceedings of the 15th
   International Workshop on the ACL2 Theorem Prover and Its
   Applications, Austin, Texas, USA, November 5-6, 2018. Electronic
   Proceedings in Theoretical Computer Science.} Matt Kaufmann and
   Shilpi Goel, editors.  Vol.  280, pp. 18-29, 2018. URL
   \url{https://cgi.cse.unsw.edu.au/~eptcs/paper.cgi?ACL22018.2}.

 {\bf Binary-Compatible Verification of Filesystems with ACL2}. In: \textit{10th International Conference
on Interactive Theorem Proving (ITP 2019) (Leibniz International
Proceedings in Informatics (LIPIcs))}, John Harrison, John O'Leary,
 and Andrew Tolmach (Eds.), Vol. 141. Schloss Dagstuhl-Leibniz-Zentrum
 fuer Informatik, Dagstuhl, Germany, 25:1-25:18. URL
 \url{https://doi.org/10.4230/LIPIcs.ITP.2019.25}.

\section{Coursework (selected graduate courses)}
\begin{tabular}{l p{3in}}
  \underline{UT Austin:} Automated Logical Reasoning, Introduction to
  Mathematical Logic, \\ Formal Verification and Semantics, Automatic
  Verification of Software,\\ Numerical Linear Algebra, Dependable
  Computing Sytems, \\Advanced Operating Systems, Recursion and Induction. \\
  \underline{IIT Delhi:} Compiler Design,
  Theory of Computation, Numerical Optimisation.
\end{tabular}

\section{Teaching assistantships (UT Austin)}
\begin{tabular}{l p{3in}}
  \underline{Graduate courses:} \\
  CS386L Programming Languages (Fall 2016, Spring 2020) \\
  Convex Optimization (Fall 2019) \\
  \underline{Undergraduate Courses:} \\
  CS439N Operating Systems (Fall 2015, Spring 2016) \\
  CS340D Debugging and Verifying Programs (Spring 2018) \\
  CS392F Automated Software Design (Spring 2019)
\end{tabular}

% Tabulate Technical Skills; p{3in} defines paragraph 3 inches wide
\section{Technical \\ Skills}
   \begin{tabular}{l p{3in}}
    \underline{Theorem provers:} ACL2, Coq, TLAPS. \\

    \underline{Programming languages:} Functional languages (OCaml,
    SML), \\ logic programming languages (Prolog),
    hardware description languages (VHDL, Verilog). \\

    \underline{Operating systems:} GNU/Linux (kernel and application development). \\

    \underline{Compiler frameworks:} Soot (Java), LLVM (C++). \\

    \underline{Others:} Xilinx, Matlab, PostgreSQL. \\
 \end{tabular}

\section{Scholastic Achievements}
 \begin{itemize} \itemsep -2pt  % reduce space between items
 \item Awarded the UT Austin Graduate School's Recruitment
   Fellowship. \hfill (2014-2017)
 \item All India Rank 138 (out of 400000), Joint Entrance Examination
   (IIT-JEE). \hfill (2009)
 \item Secured All India Rank 29 in the All India Engineering Entrance
   Examination (AIEEE) among 1000000 candidates.\hfill (2009)
 \item Scored 99 percentile in Verbal and Analytical Reasoning,
   GRE. \hfill (2012)
 \end{itemize}

\section{Others}
\begin{tabular}{l p{3in}}
  \underline{Languages:} English, French, Gujarati, Hindi. \\
\end{tabular}

\end{resume}
\end{document}
